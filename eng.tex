\documentclass[a4paper, 12pt]{article}
\usepackage[utf8]{inputenc}
\usepackage[T2A]{fontenc}
\usepackage{amsmath,amssymb,amsthm}
\usepackage[a4paper,hmargin=2.5cm,vmargin=2.5cm]{geometry}
\usepackage[english,russian]{babel}

\newtheorem{proposition}{Proposition}
\newtheorem{lemma}{Lemma}
\newtheorem{corollary}{Corollary}
\newtheorem{statement}{Statement}
\newtheorem{theorem}{Theorem}

\theoremstyle{definition}
\newtheorem{definition}{Definition}
\newtheorem{example}{Example}

\newtheorem{property}{Property}

\theoremstyle{remark}
\newtheorem{remark}{Remark}

% \newenvironment{pf}{\noindent\textbf{Proof.} ~ \par}{\qed}
\newenvironment{pf}{\noindent\textbf{Proof.}}{\qed}
\newcommand{\define}[1]{{\textit{#1}}}

\renewcommand{\leq}{\leqslant}
\renewcommand{\geq}{\geqslant}

\title{Approximate Homological Version of Quillen-McCord Theorem}
\author{Vitalii Guzeev, Anton Ayzenberg}

\begin{document}

\section{Abstract}

Combinatorial topology has recently been successfully applied to data analysis. One approach is to compute homology groups of some persistence complex constructed by experimental data and try to classify some objects by them. However complexity of computation of persistent homology is high enough to limit practical usage of the approach.\\

There is a question of how to construct persistent complex with same homology as initially given with smaller dimension. In this paper we prove Quillen-McCord theorem (Quillen fiber lemma) in the setting of persistent homology. It can possibly be used to reduce persistence complex to smaller one if given complex is equivalent to a nerve of some partially ordered set.

\section{Preliminaries}

\subsection{Persistence modules and $\varepsilon$-interleavings}

Basic definition is the following:

\begin{definition}
  \define{Persistence complex} is a family of chain complexes $C_{\star}^{i_0} \xrightarrow{f_{i_0}} C_{\star}^{i_1} \xrightarrow{f_{i_1}} C_{\star}^{i_2} \xrightarrow{f_{i_2}} \ldots$ with $I$ --- some indexing set and $f_i$ being chain maps.
\end{definition}

These objects naturally arise in experiments. If we consider $I$ as time, this is a structure we obtain while observing some dynamic structure which can at any time be represented by a chain complex.\\

\begin{definition}
  Let $R$ be a ring. \define{Persistence module} is a family of $R$-modules $M^i$ with homomorpisms $\phi_i :: M^i \to M^{i+1}$.
\end{definition}

Example of a persistent module is given by homology modules of persistence complex $C_{\star}$ (\define{persistent homology}). $H_i^j(C_{\star}) = H_i(C_{\star{j}})$, maps $\phi_j$ are induced by $f_i$.\\

\begin{definition}
  Persistence complex (module) is of \define{finite type} over $R$ if all component complexes (modules) are finitely-generated as $R$-modules and all $f_i$ ($\phi_i$) are isomorphisms for $i > m$ for some $m$.
\end{definition}

Since experiments usually take finite time and operate finite amount of data, we can safely consider only complexes of finite type. Note that by construction homology of complex of finite type is a module of finite type.\\

\begin{definition}
  \define{Graded module} over $I$-graded ring $R$ with graded components $R_i$ is an $R$-module $M$ together with a decomposition $M = \bigoplus_{j \in I} M^j:\; \forall i \in I\, R_i \cdot M^j \subset M^{j+i}$. For correctness of the definition it's enough for $I$ to be a semigroup.
\end{definition}

% Why only linear?
\begin{definition}
  Assume $I$ is a monoid with linear order. Then \define{non-negatively graded module} over $I$-graded ring $R$ is an $R$-module $M$ together with a decomposition $M = \bigoplus_{j \in I} M^j:\; \forall i \in I_{\geq 0}\, R_i \cdot M^j \subset M^{j+i}$
\end{definition}

\begin{proposition}. The following hold:
  \begin{enumerate}
    \item Graded modules over $R$ form a category.
    \item Non-negatively graded modules over $R$ form a category.
  \end{enumerate}
\end{proposition}

Morphisms in both these categories are morphisms $\phi$ between modules $M$ and $N$ such that $\phi(M^j) \subset N^j$ for all $j$ in indexing set.\\

\begin{theorem} {\cite[Theorem 3.1]{Zomorodian05}}\\
  Category of persistence modules of finite type over $R$ is equivalent to category of non-negatively graded finitely generated $R[t]$-modules.
\end{theorem}

\begin{remark}
  If $R$ is a field then $R[t]$ is a PID and graded ideals are exactly $t^n$ for $n \in I_{\geq 0}$.
\end{remark}

% Continuous time?

We have introduced a setting to work with persistence complexes as with graded modules. However computations in a category of non-negatively graded modules are not fit for applications since experiment always comes with an error and we do not have a measure for similarity of two persistence modules. We need technique which allows to construct morphisms of deformed complexes and track error propagation. This technique is developed in {\cite{GS16}}, we will recall it without proofs.

\begin{definition} {\cite[Definition 2.7]{GS16}}\\
  Let $M$ and $N$ be persistent modules, $f : M \to N$ be a homomorphism of modules. Then $f$ is called \define{$\varepsilon$-morphism} if $f(M^j) \subset N^{j+\varepsilon}$ for $\varepsilon \geq 0$.
\end{definition}

\begin{remark}
  Note that $0$-morphism is a morphism in category of non-negatively graded modules over $R[t]$.
\end{remark}

\begin{remark}
  Let $R$ be a field. Then for every persistence module $M$ and any $\varepsilon$ there exists morphism $Id_{\varepsilon}$ such that $Id_{\varepsilon}(a) = t^{\varepsilon}a$.
\end{remark}

\begin{remark}
  Restriction on $R$ to be a field is crucial for construction. It is a strong restriction, however, it is acceptable.
\end{remark}

\begin{definition}
  Persistence modules $M$ and $N$ are called \define{$\varepsilon$-interleaved} ($M \stackrel{\varepsilon}{\sim} N$) if there exist pair of $\varepsilon$-morphisms $(\phi : M \to N,\;\psi : N \to M)$ (\define{$\varepsilon$-interleaving}) such that $\phi \circ \psi = Id_{2\varepsilon} : N \to N$ and $\psi \circ \phi = Id_{2\varepsilon} : M \to M$.
\end{definition}

\begin{remark}
There follows that $M \stackrel{\varepsilon}{\sim} N$ implies $M \stackrel{\alpha}{\sim} N$ for any $\alpha > \varepsilon$ since for $\varepsilon$-interleaving $(\phi, \psi)$ we have $\alpha$-interleaving $(Id_{\alpha - \varepsilon} \circ \phi, Id_{\alpha - \varepsilon} \circ \psi)$.
\end{remark}

\begin{definition}
  We denote as \define{$\varepsilon$-equivalence} relation with the following properties:\\
  For any $M$, $N$, $L$, $\varepsilon$, $\varepsilon_1$, $\varepsilon_2$
  \begin{itemize}
    \item $M \stackrel{0}{\sim} M$.
    \item $M \stackrel{\varepsilon}{\sim} N$ is equivalent to $N \stackrel{\varepsilon}{\sim} M$.
    \item if $M \stackrel{\varepsilon_1}{\sim} N$ and $N \stackrel{\varepsilon_2}{\sim} L$ then $M \stackrel{\varepsilon_1 + \varepsilon_2}{\sim} L$.
  \end{itemize}
\end{definition}

\begin{proposition}
  $\varepsilon$-interleaved persistence modules are $\varepsilon$-equivalent.
\end{proposition}

\begin{definition}
  Persistent module $M$ is called \define{$\varepsilon$-trivial} if $M \stackrel{\varepsilon}{\sim} 0$. It is equivalent to $t^{2\varepsilon}M = 0$.
\end{definition}

\begin{lemma}
  Let $0 \to M \to L \to N \to 0$ be a short exact sequence. Then the following properties hold:
  \begin{itemize}
    \item If $M \stackrel{\varepsilon_1}{\sim} 0$ and $N \stackrel{\varepsilon_2}{\sim} 0$ then $L \stackrel{\varepsilon_1 + \varepsilon_2}{\sim} 0$.
    \item If $L \stackrel{\varepsilon}{\sim} 0$ then $M \stackrel{\varepsilon}{\sim} 0$ and $N \stackrel{\varepsilon}{\sim} 0$.
  \end{itemize}
\end{lemma}

\subsection{Quillen-McCord theorem}

We give a formulation of classical theorem and then descend from general categorical case to specific.\\

\begin{definition}
  \define{Simplex category} $\Delta$ is the category with objects --- nonempty linearly ordered sets of the form $[n] = \{0,1,\ldots,n\}$ with $n \geq 0$ and morphisms --- order-preserving functions.
\end{definition}

All morphisms in the simplex category are compositions of injective maps $\delta_i^n : \{0,1,\ldots,i-1,i+1\ldots,n\} \mapsto \{0,1,\ldots,i-1,i,i+1,\ldots,n\}$ and surjective maps $\sigma^n_i : \{0,1,\ldots,i,i+1,i+2\ldots,n\} \mapsto \{0,1,\ldots,i,i+1,\ldots,n\}$.

\begin{definition}
  \define{Simplicial set} $S$ is a contravariant functor $\Delta \to Set$. Objects of image of $S$ are called simplices, cardinality of a simplex is called a dimension of a simplex. We will identify $s$ with set of its simplices while keeping in mind general structure.
\end{definition}

Simplicial sets form a category under natural transformations (category of presheaves on $\Delta$).\\

\begin{definition}
  Simplicial set $S$ is called a \define{simplicial complex} if for any $s \in S$ s.t. $v \subset s$  $v \in s$ and $S$ is faithful.
\end{definition}

We will also use the following definitions:

\begin{definition}
  \define{Join} $A \star B$ of simplicial complexes $A$ and $B$ is the simplicial complex with simplices --- all possible unions of simplices $a \in A$ and $b \in B$.
\end{definition}

\begin{definition}
  \define{Join} of topological spaces $A$ and $B$ is defined as follows: $A \star B\ := A \sqcup_{p_0} (A \times B \times [0,1]) \sqcup_{p_1} B$, where $p$ are projections of the cylinder $A \times B \times [0,1]$ onto faces.
\end{definition}

\begin{definition}
  \define{Star} $\operatorname{st}(x)$ of simplex $x \in A$ ($A$ simplicial complex) is the simplicial complex with simplices --- all simplices $a \in A$ such that there exists inclusion $x \hookrightarrow a$.
\end{definition}

\begin{definition}
  \define{Link} $\operatorname{lk}(x)$ of simplex $x \in A$ is defined as follows: $\operatorname{lk}(x) = \{v \in \operatorname{st}(x)|\; x \not\in v\}$.
\end{definition}

\begin{definition}
  Functor $\mathcal{K} : \Delta \to Top$ which maps simplices to geometric simplices of corresponding dimension and morphisms to inclusions of faces and restrictions to subcomplexes is called \define{geometric realization}.
\end{definition}

\begin{proposition}
  $\operatorname{st}(x) = \operatorname{lk}(x) \star x$. Hence $\mathcal{K}(\operatorname{st}(x))$ is a cone over $\mathcal{K}(x)$.
\end{proposition}

\begin{proposition}
  $\mathcal{K}(A \star B) = \mathcal{K}(A) \star \mathcal{K}(B)$
\end{proposition}

Let $C$ denote a category. We can construct the simplicial set called \define{nerve of category $C$} as follows:\\
Let objects of $C$ be the only 0-dimensional simplices. Then let all morphisms be 1-dimensional simplices, all composable pairs of morphisms be 2-dimensional simplices and so on.\\
This construction is functorial, we denote nerve functor as $\mathcal(N)$.

\begin{definition}
  Geometric realization $BC$ of $\mathcal{N}(C)$ is called \define{classifying space} of $C$.\\

  We will denote composition of nerve and geometric realization as $\mathcal{B}$. It is obviously a functor and we only need this construction to avoid confusing notation $Bf$ for induced morphism of classifying spaces. By definition $\mathcal{B}(C) = BC$.
\end{definition}

\begin{definition}
  Let $f: C \to D$ be a functor and $d$ --- object in $D$. Then \define{comma category} $d \downarrow f$ is a category with objects --- pairs $(s,i_s)$ of objects in $C$ and morphisms $i_s : d \to f(s)$ and morphisms --- morphisms $g$ in $C$ such that triangle $i_s, i_{g(s)}, f(g)$ commutes.
\end{definition}

\begin{theorem} {\cite[Theorem A]{Quillen72}}\\
  If $f: C \to D$ is a functor such that the classifying space $B(d \downarrow f)$ of the comma category $d \downarrow f$ is contractible for any object $d \in D$, then $f$ induces a homotopy equivalence $BC \to BD$.
\end{theorem}

Every poset $X$ with order relation $R$ can be seen as a category if we set $\operatorname{Ob}(X) = X$ and $Hom_X(a,b) = \{r_{ab}\}$ if $(a,b) \in R$ and $\emptyset$ otherwise. Note that map between posets is functorial if and only if it preserves order.\\
Nerve construction on a poset yields a simplicial complex called \define{order complex}. Application of Quillen A theorem to posets yields the following theorem:

\begin{theorem} \textbf{Quillen-McCord theorem}\\
  Assume $X, Y$ are finite posets, $f : X \to Y$ is order-preserving map.\\
  If $\forall y \in Y\;\mathcal{B}(f^{-1}(Y_{\leqslant y}))$ is contractible, then $\mathcal{B}f$ is a homotopy equivalence between $BX$ and $BY$.\\
\end{theorem}

There holds homological {\cite[Corollary 5.5]{Bar11}} versions of this theorem:

\begin{theorem} \textbf{Homological Quillen-McCord theorem}\\
  Assume $X, Y$ are finite posets, $f : X \to Y$ is order-preserving map, $R$ is a PID.\\
  If $\forall y \in Y\;H_i(\mathcal{B}(f^{-1}(Y_{\leqslant y})),R) = 0$ for any $i$, $\mathcal{B}f$ induces isomorphisms of all homology groups with coefficients in $R$ on $BX$ and $BY$.\\
\end{theorem}

Proofs of both theorems are necessary for construction of a desired result and will be recalled in section 2.4 as well as a specific proof of Quillen-McCord theorem {\cite[Proof of Theorem 1.1]{Bar11}}.\\

\subsection{Statement of the theorem}

Assume we conduct an experiment and observe dynamics of a simplicial complex $S^{i_0} \xrightarrow{f_{i_0}} S^{i_1} \xrightarrow{f_{i_1}} S^{i_2} \xrightarrow{f_{i_2}} \ldots$. Denote the whole sequence as $S$. For each simplicial complex $S^{i}$ we can assign chain complex $C_{\star}^{i}$ via standard construction of simplicial chain complex used in the definition of simplicial homology.\\

I.e. we observe persistence chain complex $C_{\star}^{i_0} \xrightarrow{f'_{i_0}} C_{\star}^{i_1} \xrightarrow{f'_{i_1}} C_{\star}^{i_2} \xrightarrow{f'_{i_2}} \ldots$. Each simplicial complex in a series has geometric realization with the same homology as its chain complex.

\begin{definition}
  We will denote as \define{persistence topological space} $X$ sequence of topological spaces $X^{i_0} \xrightarrow{f_{i_0}} X^{i_1} \xrightarrow{f_{i_1}} X^{i_2} \xrightarrow{f_{i_2}} \ldots$ with compatible triangulations (chain complexes formed by simplicial complexes) $C_{\star}^{i}$ such that $C_{\star}$ is a persistence complex with structure maps induced by $f_i$. Homology of $X$ is a persistence module with components --- homology modules of $X^i$.\\

  We will denote as morphism $g$ between persistence topological spaces $X$ and $Y$ collection of maps $g_i : X^i \to Y^i$ between components such that all $g_i$ commute with $f_i$.
\end{definition}

Apparently $\mathcal{K}S$ is a persistence topological space.\\

\begin{definition}
  Persistence topological space $X$ is called \define{$\varepsilon$-acyclic} over $R$ if for all indices\\ $H_i(X,R) \stackrel{\varepsilon}{\sim} 0$.
\end{definition}

\begin{definition}
  Consider pair of morphisms $(f : X \to Y, g : Y \to X)$ between persistence topological spaces. It is called homological $\varepsilon$-equivalence if induced maps on graded homology modules form $\varepsilon$-interleaving.
\end{definition}

To formulate a theorem we have to move to terms of posets. Define \define{persistence poset} as series of posets with structure maps analogously to persistence complexes and map of persistence posets as collection of maps between posets commuting with structure maps. It induces map on persistence order complexes. We also analogously define persistence poset of finite type --- it is a finite sequence of finite posets. Easy to see that nerve (as a functor) of persistence poset of finite type is a persistence complex of finite type.\\
We also need some restricting technical conditions on structure maps of persistent order complex. We will specify them during the proof since they are the major limiting factor in applications of the theorem.

\begin{theorem} \textbf{Approximate Quillen-McCord theorem, draft statement}\\
  Assume $X, Y$ are persistent posets of finite type with structure maps satisfying additional conditions, $f : X \to Y$ is order-preserving map, $R$ is a field or $\mathbb{Z}$.\\
  If $\forall y \in Y\;\mathcal{B}(f^{-1}(Y_{\leqslant y}))$ is $\varepsilon$-acyclic over $R$, $\mathcal{B}f$ induces $e$-interleavings of all homology modules over $R$ on $BX$ and $BY$.\\
\end{theorem}

Value of $e$ will also be specified during the proof and is critical to applications being measure of error of approximation.

\subsection{Prerequisite results}

\subsubsection{Quillen-McCord theorem}

This section contains the proof of Quillen-McCord theorem given by Barmak.

\begin{proposition} {\cite[Proposition 2.1]{Bar11}}
  Let $K_1$ and $K_2$ be two geometric simplicial complexes and geometric simplicial complex $K = K_1 \cup K_2$ (as space and simplicial complex). Then if $K_1 \cup K_2 \hookrightarrow K_1$ is a homotopy equivalence, then so is $K_2 \hookrightarrow K$.
\end{proposition}

\begin{proposition} {\cite[Proposition 2.2]{Bar11}}
  Let $f,g : X \to Y$ be order-preserving maps between finite posets such that $\forall x\;f(x) \leq g(x)$. Then $\mathcal{B}(f)$ is homotopy-equivalent to $\mathcal{B}(g)$.
\end{proposition}

\begin{pf}
  $X$ has a finite set $M$ of maximal elements. Take any of them ($m_1$) and define $h_1 : X \to Y$ such that $h_1(m_1) = g(m_1)$ and $h_1 = f$ on all other elements of $X$. This is an order-preserving map due to maximality of $m_1$. Take $M \setminus \{m_1\}$ and build $h_2$ which is equal to $h_1$ on complement to $m_2$ and to $g$ on $m_2$ and so on. We have built a finite sequence of maps $h_0 = f \leq h_1 \leq h_1 \leq \ldots \leq g = h_n$.\\

  Elements $h_i(m_i)$ and $h_{i-1}(m_i)$ are comparable. Hence there exists simplex $\{h_i(m_i); h_{i-1}(m_i)\}$ in $\mathcal{N}(Y)$. Since $mathcal{N}(Y)$ is a simplicial complex, for other elements between selected where exist no holes and thus where is a linear homotopy between $h_{i-1}$ and $h_{i}$ which contracts simplex $\{h_i(m_i); h_{i-1}(m_i)\}$.\\

  Hence there is a homotopy between $f$ and $g$.
\end{pf}

\begin{proposition}
  Note that $\operatorname{lk}(\mathcal{N}(x)) = \mathcal{N}(X_{>x}) \star \mathcal{N}(X_{<x})$. Therefore $\mathcal{K}(\operatorname{lk}(\mathcal{N}(x))) = \mathcal{B}(X_{>x}) \star \mathcal{B}(X_{<x})$.
\end{proposition}

\begin{remark}
  $\mathcal{B}(X) = \mathcal{B}(X \setminus \{x\}) \cup \mathcal{K}(\operatorname{st}(\mathcal{N}(x)))$ for any $x \in X$.\\
\end{remark}

\begin{lemma}
  Let $X$ be a finite poset and for $x \in X$ either $\mathcal{B}(X_{>x})$ or $\mathcal{B}(X_{<x})$ is contractible. Then embedding $\mathcal{B}(X \setminus \{x\}) \hookrightarrow \mathcal{B}(X)$ is a homotopy equivalence.
\end{lemma}

\begin{pf}
  By proposition $\mathcal{K}(\operatorname{lk}(\mathcal{N}(x)))$ is contractible. Hence its embedding to its cone $\mathcal{K}(\operatorname{st}(\mathcal{N}(x)))$ is homotopy equivalence. Lemma follows by Proposition 5.
\end{pf}

\begin{definition}
  Let $f : X \to Y$ be an order preserving map between posets. Denote orders ($\leq$) on $X$ and $Y$ as $R_X$ and $R_Y$. Then we define poset $M(f) = X \cup Y$ with $R = R_X \cup R_Y \cup R_{f}$ where $(x,y) \in R_f$ if and only if $(f(x),y) \in R_Y$.\\

  We will by analogy denote this poset a \define{mapping cylinder} of $f$. There are also defined canonical inslusions $i : X \to M(f)$ and $j : Y \to M(f)$.
\end{definition}

\begin{pf} \textbf{Quillen-McCord theorem}\\
  Let $X, Y$ be finite posets with order-preserving map $f : X \to Y$.\\

  Every poset has linear extension. Let $x_1, x_2, \ldots, x_n$ be enumeration of $X$ in such linear order and $Y^r = \{x_1,\ldots,x_r\} \cup Y \subset M(f)$ for any $r$.\\

  Consider $Y^r_{>x_r} = Y_{\geq f(x_r)}$. $\mathcal{B}(Y_{\geq f(x_r)})$ is a cone over $\mathcal{B}(f(x_r))$. It is contractible, therefore $\mathcal{B}(Y^{r-1}) \hookrightarrow \mathcal{B}(Y^{r})$ is homotopy equivalence by lemma. By iteration $\mathcal{B}(j) : \mathcal{B}(Y^{0}) = \mathcal{B}(Y) \hookrightarrow \mathcal{B}(M(f)) = \mathcal{B}(Y^n)$ is homotopy equivalence between $BY$ and $M(f)$.\\

  Then consider linear extension of $Y$ with enumeration $y_1,\ldots,y_m$ and $X^r = X \cup \{y_{r+1},\ldots,y_n\} \subset M(f)$. $X^{r-1}_{\leq y_r} = f^{-1}(Y_{\leqslant y_r})$. Latter is contractible by condition of the theorem. Hence $\mathcal{B}(X^{r}) \hookrightarrow \mathcal{B}(X^{r-1})$ is homotopy equivalence and $\mathcal{B}(i)$ is a homotopy equivalence between $X$ and $M(f)$.\\

  Note that $i(x) \leqslant (j \circ f)(x)$. By Proposition 6 $\mathcal{B}(i)$ is homotopic to $\mathcal{B}(j \circ f) = \mathcal{B}(j) \circ \mathcal{B}(f)$. Hence $\mathcal{B}(f)$ is the homotopy equivalence between $BX$ and $BY$.
\end{pf}

\subsubsection{Homological Quillen-McCord theorem}

To derive homological version of the theorem we can variate the proof of standard version in the following manner:\\

\begin{proposition} {\cite[Lemma 2.1]{Milnor56}}
  Reduced homology groups with coefficients in a principal ideal domain of a join satisfy the relation
  $H_{r+1}(A \star B, R) \simeq \bigoplus_{i+j=r}(H_i(A,R) \otimes_R H_j(B,R)) \oplus \bigoplus_{i+j=r-1} \operatorname{Tor_1^R}(H_i(A,R),H_j(B,R))$.
\end{proposition}

\begin{lemma}
  Let $X$ be a finite poset and for $x \in X$ either $H_i(\mathcal{B}(X_{< x}))$ or $H_i(\mathcal{B}(X_{> x}))$ with coefficients in a PID are trivial for any $i$. Then embedding $\mathcal{B}(X \setminus \{x\}) \hookrightarrow \mathcal{B}(X)$ induces isomorphisms of all homology groups.
\end{lemma}

\begin{pf} ~ \par
  By Proposition 8 $H_i(\mathcal{K}(\operatorname{lk}(\mathcal{N}(x)))) = H_i(\mathcal{B}(X_{>x}) \star \mathcal{B}(X_{<x}))$ are trivial for all indices $i$ --- $Tor$-functors vanish if any of their arguments is trivial.
  Application of Mayer-Vietoris long exact sequence to covering from Remark 6 yields the lemma.
\end{pf}\\

Proof of the theorem is similar to proof finishing previous section. However, we write it here in detail in order to be able to highlight differences. Changed parts are written in italic.\\

\begin{pf} \textbf{Homological Quillen-McCord theorem}\\
Let $X, Y$ be finite posets with order-preserving map $f : X \to Y$.\\

Every poset has linear extension. Let $x_1, x_2, \ldots, x_n$ be enumeration of $X$ in such linear order and $Y^r = \{x_1,\ldots,x_r\} \cup Y \subset M(f)$ for any $r$.\\

Consider $Y^r_{>x_r} = Y_{\geq f(x_r)}$. $\mathcal{B}(Y_{\geq f(x_r)})$ is a cone over $\mathcal{B}(f(x_r))$. It is contractible, therefore $\mathcal{B}(Y^{r-1}) \hookrightarrow \mathcal{B}(Y^{r})$ is homotopy equivalence by lemma 2. By iteration $\mathcal{B}(j) : \mathcal{B}(Y^{0}) = \mathcal{B}(Y) \hookrightarrow \mathcal{B}(M(f)) = \mathcal{B}(Y^n)$ is homotopy equivalence between $BY$ and $M(f)$.\\

Then consider linear extension of $Y$ with enumeration $y_1,\ldots,y_m$ and $X^r = X \cup \{y_{r+1},\ldots,y_n\} \subset M(f)$. $X^{r-1}_{\leq y_r} = f^{-1}(Y_{\leqslant y_r})$. \textit{Latter is acyclic over $R$ by condition of the theorem. Hence $\mathcal{B}(X^{r}) \hookrightarrow \mathcal{B}(X^{r-1})$ induces isomorphisms of all homology groups and by functoriality of homology $\mathcal{B}(i)$ induces isomorphisms of all homology groups between $X$ and $M(f)$.}\\

Note that $i(x) \leqslant (j \circ f)(x)$. By Proposition 6 $\mathcal{B}(i)$ is homotopic to $\mathcal{B}(j \circ f) = \mathcal{B}(j) \circ \mathcal{B}(f)$. \textit{Homotopic maps induce same maps on homology, $j$ is a homotopy equivalence and induce isomorphisms. Hence $\mathcal{B}(f)$ induce isomorphisms between $H_i(BX,R)$ and $H_i(BY,R)$.}
\end{pf}\\

We see two updates. First one is essential, it requires Lemma 3 and operates some equivalence propagating in a chain of length equal to cardinality of $Y$. Second follows automatically from functoriality of all used constructions, but contains statement about composition of functions. We expect these two parts to be changed while constructing a proof for approximate version of the theorem.

\subsubsection{Error propagation in spectral sequences}

\section{Results}

\section{References}

\begin{enumerate}
  \bibitem[Bar11]{Bar11}
  Jonathan Ariel Barmak. 2011.
  \newblock On Quillen’s Theorem A for posets.
  \newblock J. Comb. Theory Ser. A 118, 8 (November 2011), 2445–2453.
  \newblock DOI:https://doi.org/10.1016/j.jcta.2011.06.008
  \bibitem[Zomorodian05]{Zomorodian05}
  A. Zomorodian and G. Carlsson.
  \newblock Computing persistent homology.
  \newblock In: Discrete and Computational Geometry 33.2 (2005), pp. 249–274.
  \bibitem[GS16]{GS16}
  Govc, Dejan \& Skraba, Primoz. (2016).
  \newblock An Approximate Nerve Theorem.
  \newblock Foundations of Computational Mathematics.
  \newblock 10.1007/s10208-017-9368-6.
  \bibitem[Quillen72]{Quillen72}
  Daniel Quillen,
  \newblock Higher algebraic K-theory, I: Higher K-theories Lect.
  \newblock Notes in Math. 341 (1972), 85-1
  \bibitem[Milnor56]{Milnor56}
  Milnor, J. (1956).
  \newblock Construction of Universal Bundles, II.
  \newblock Annals of Mathematics, 63(3), second series, 430-436.
  \newblock doi:10.2307/1970012
\end{enumerate}

\subsection*{5. Доказательство теоремы 3}

Приведём здесь утверждение леммы 2.1 из [5]:\\
\begin{lemma}
  $H_{r+1}(A * B) \simeq \bigoplus_{i+j=r}(H_i(A) \otimes H_j(B)) \oplus \bigoplus_{i+j=r-1} \operatorname{Tor}(H_i(A),H_j(B))$.\\
  Гомологии здесь рассматриваются приведённые с коэффициентами в области главных идеалов.
\end{lemma}

Аналог леммы 4 разбивается на два независимо доказываемых утверждения.\\

\begin{lemma}
  Пусть $X$ --- конечное частично упорядоченное множество, $x \in X$ --- элемент такой, что или $\mathcal{K}(X_{>x})$, или $\mathcal{K}(X_{<x})$ $m$-ацикличен.\\
  Тогда $lk(x) = \mathcal{K}(X_{>x}) * \mathcal{K}(X_{<x})$ $m$-ацикличен.
\end{lemma}
\begin{proof}
  Пусть для определённости в лемме 7 $B$ выбран $m$-ацикличным. Этот выбор произволен в силу коммутативности операции джойна.\\
  По линейности тензорного произведения из $m$-ацикличности $B$ следует $m$-тривиальность $H_i(A) \otimes H_j(B)$ для любых индексов.\\
  Из леммы 2 подмодули и фактормодули $m$-тривиальных модулей $m$-тривиальны.\\
  В комплексе $\ldots \to P_n \otimes B \to \ldots \to P_1 \otimes B \to 0$ все модули $m$-тривиальны, следовательно, все его гомологии, то есть $Tor_{\bullet}(A,B)$ $m$-тривиальны.\\
  Прямая сумма $m$-тривиальных модулей $m$-тривиальна, $m$-перемежения строятся покомпонентно. Следовательно, правая сторона выражения леммы 7 $m$-тривиальна, а значит, $m$-тривиальная и левая сторона, что и требовалось.
\end{proof}

\begin{lemma}
  Пусть $A \to B \xrightarrow{f} C \to D$ --- подпоследовательность точной последовательности, $A$ и $D$ $m$-тривиальны. Тогда $B$ и $C$ $2m$-перемежены.
\end{lemma}
\begin{proof}
  Из точности следует, что $\operatorname{Ker}(f)$ и $\operatorname{coKer}(f)$ $m$-тривиальны, следовательно, по приближённой транзитивности $2m$-перемежены. При этом $\operatorname{coIm}(f) \stackrel{f}{\simeq} \operatorname{Im}(f)$.\\
  $B = \operatorname{Ker}(f) \oplus \operatorname{coIm}(f)$; $C = \operatorname{coKer}(f) \oplus \operatorname{Im}(f)$.\\
  $2m$-перемежение $B$ и $C$ строится покомпонентно.
\end{proof}

\begin{lemma}
  \textit{Приближённый вариант леммы 4}\\
  Пусть $X$ --- конечное частично упорядоченное множество, $x \in X$ --- элемент такой, что или $\mathcal{K}(X_{>x})$, или $\mathcal{K}(X_{<x})$ $m$-ацикличен. Тогда вложение $\mathcal{K}(X\setminus \{x\}) \hookrightarrow \mathcal{K}(X)$ является гомологической $2m$-эквивалентностью.
\end{lemma}
\begin{proof}
  Лемма следует из лемм 8 и 9, в качестве точной последовательности в лемме 9 можно взять точную последовательность Майера-Вьеториса для того же разбиения, что и в лемме 6.
\end{proof}

Доказанных утверждений достаточно для построения доказательства теоремы 3.

\subsection*{6. Приложение: устойчивость в спектральных последовательностях}
Мы будем рассматривать ограниченные в каком-то направлении гомологические спектральные последовательности, расположенные в первом квадранте.\\
Расположение в первом квадранте: $p < 0 \lor q < 0 \implies E_r^{p,q} = 0$.\\
Ограниченность: $\exists p_0:\;\forall p>p_0,\,\forall q\; E_r^{p,q} = 0$ или $\exists q_0:\;\forall q>q_0,\,\forall p\; E_r^{p,q} = 0$.\\
Эти ограничения означают, что на какой-то странице все дифференциалы становятся нулевыми, поскольку их образы лежат за пределами квадранта, то есть нулевые. Значит, c некоторой страницы $n$ спектральная последовательность стабилизируется, т.е. $E_n = E_{\infty}$.\\
Гомологичность в этом контексте означает, что дифференциалы уменьшают сумму индексов.\\

\begin{lemma}
  Пусть все модули на $r$-й странице спектральной последовательности $m$-тривиальны. Тогда на $r+1$-й странице все модули $m$-тривиальны.
\end{lemma}
\begin{proof}
  Каждый модуль на $r+1$-й странице является модулем гомологий некоторого модуля с $r$-й страницы относительно дифференциалов $r$-й страницы. То есть он является фактор-модулем подмодуля $m$-тривиального модуля. Применяя дважды лемму 2, получаем, что он является фактор-модулем $m$-тривиального модуля, то есть $m$-тривиален.
\end{proof}

\begin{definition}
  Будем говорить, что спектральная последовательность сходится к набору модулей $A_{\bullet}$,
  если для любого $n$ на $A_n$ определена убывающая фильтрация $\ldots \hookrightarrow  F^pA \hookrightarrow F^{p-1}A \hookrightarrow \ldots \hookrightarrow F^0A = A$ и для каждого $n$ $E^{p,n-p} = F^{p-1}A/F^pA$.
\end{definition}

Заметим, что по определению ограниченная спектральная последовательность в первом квадранте может сходится только к набору модулей, на каждом из которых определена фильтрация с конечным числом ненулевых членов, что будем записывать как $0 = F^nA \hookrightarrow F^{n-1}A \hookrightarrow \ldots \hookrightarrow F^0A = A$.

\begin{lemma}
  Пусть $0 = F^nA \hookrightarrow F^{n-1}A \hookrightarrow \ldots \hookrightarrow F^0A = A$ --- фильтрация, при этом $\forall p\; F^pA/F^{p+1}A$ $m$-тривиален.\\
  Тогда $A$ $nm$-тривиален.
\end{lemma}

\begin{proof}
  $F^{n-1}A = F^{n-1}A/F^nA$, следовательно, $m$-тривиален.\\
  Далее индукция по номеру члена фильтрации.\\
  Пусть $F^{n-p}A$ $pm$-тривиален. $F^{n-(p+1)}A/F^{n-p}A$ $m$-тривиален по условию. Рассмотрим точную тройку $0 \to F^{n-p} \hookrightarrow F^{n-(p+1)} \rightarrow F^{n-(p+1)}A/F^{n-p}A \to 0$. По лемме 1 $F^{n-(p+1)}$ $pm+m=(p+1)m$-тривиален, что доказывает шаг индукции.\\
\end{proof}

Следствием Леммы 4 является следующее общее утверждение:
\begin{corollary}
  Если все модули на первой странице спектральной последовательности $m$-тривиальны и последовательность стабилизируется на $n$-й странице, последовательность сходится к набору $nm$-тривиальных модулей.
\end{corollary}

\end{document}
