\documentclass[a4paper,12pt]{report}
\usepackage{tabularx}
\usepackage{graphicx}
\usepackage{geometry}
\usepackage{wrapfig}
\usepackage{amsfonts, amsmath, amssymb, mathtools, faktor, amsthm}
\usepackage{fancybox,fancyhdr}
\usepackage{lastpage, hyperref}
%\geometry{a5paper,width=120mm,height=190mm,heightrounded,footskip=39pt,marginratio={1:1},includeheadfoot}

\usepackage{mdwtab}\arrayextrasep0pt % êîìàíäà \hlx
%\usepackage{mytab} % ìàêðîñû \settab è \starttab

\usepackage[utf8]{inputenc}
\usepackage[russian]{babel}

\usepackage{enumerate}
\usepackage{tikz-cd}

%\renewcommand{\baselinestretch}{0.8}

\medskipamount7.8pt %plus 3pt

\leftskip15mm
\parskip0pt

\emergencystretch2.5em

\relpenalty10000
\binoppenalty10000
\postdisplaypenalty10000
\sloppy

\setlength{\topmargin}{-1cm}
\setlength{\textheight}{237mm}
\setlength{\textwidth}{162mm}
\setlength{\oddsidemargin}{-0.9cm}
%\topmargin, \oddsidemargin, \evensidemargin, \textwidth. \textheight

\parindent=1cm

\hyphenpenalty10000

\medmuskip3mu minus 1mu
\thickmuskip3mu minus 1mu
\thinmuskip 5mu minus2mu

\everydisplay{\thickmuskip5mu}

\def\hm#1{#1\nobreak\discretionary{}{\hbox{$#1$}}{}}

%\renewcommand{\theenumi}{\alph{enumi}}
%\renewcommand{\labelenumi}{\theenumi)}

\newtheorem{theorem}{Теорема}
\newtheorem{lemma}{Лемма}
\newtheorem{claim}{Утверждение}
\newtheorem{definition}{Определение}
\newtheorem{definition*}{Определение*}

\begin{document}

	\section*{Черновик курсовой: Приближённая гомологическая версия теоремы Квиллена-МакКорда.}
  \textit{Виталий Гузеев, 2019-2020, Москва}

  \hrulefill

\subsection*{1. Необходимые определения}
В этом техническом разделе помещён некоторый список определений, далее в тексте по секциям предполагаемых известными. При чтении его можно пропустить. Курсивом помечены ремарки, полезные для формирования общей картины, но не используемые в работе как таковой, звёздочкой помечены требуемые для них определения. Предполагается, что читатель знаком с базовыми понятиями алгебры, топологии и геометрии, а также с теорией множеств.\\
\paragraph*{2:}

\begin{definition}
Класс $\operatorname{Ob}(\mathcal{C})$ --- множество объектов такое, что для каждой пары $A,B \in \operatorname{Ob}(\mathcal{C})$ определено множество $\operatorname{Hom}(A,B)$ морфизмов из $A$ в $B$, называется предкатегорией.
\end{definition}

\begin{definition}
Предкатегория, в которой для каждой пары морфизмов определена композиция ($\phi \in \operatorname{Hom}(A,B),\; \psi \in \operatorname{Hom}(B,C) \implies \psi \circ \phi \in \operatorname{Hom}(A,C)$) и для каждого объекта определён тождественный морфизм
($id \in \operatorname{Hom}(A,A):\; \phi \in \operatorname{Hom}(B,A) \implies id \circ \phi = \phi,\; \psi \in \operatorname{Hom}(A,B) \implies \psi \circ id = \psi$ для любых $B$ и морфизмов $\phi$, $\psi$) называется категорией.\\
\end{definition}

\begin{definition}
Отображение $\mathcal{F} :: \mathcal{A} \to \mathcal{B}$ между категориями такое, что $\forall A:\;\mathcal{F}(Id_A) = Id_{\mathcal{F}(A)}$,
$\mathcal{F}(\phi \circ \psi) = \mathcal{F}(\phi) \circ \mathcal{F}(\psi)$ называется функтором.
\end{definition}

\begin{definition}
Симплициальным комплексом называется топологическое пространство, представленное как объединение множеств, гомеоморфных стандартным симплексам.
\end{definition}

\begin{definition*}
Категория, класс объектов которой является множеством, называется малой.
\end{definition*}

\begin{definition*}
Нервом малой категории будем называть частично упорядоченное множество, элементы которого соотвествуют объектам категории ($\mathcal{N}(A)$), а пара($\mathcal{N}(A),\mathcal{N}(B)$) входит в отношение порядка тогда и только тогда, когда $\operatorname{Hom}(A,B) \neq \emptyset$.
\end{definition*}

\subsection*{2. Постановка задачи}
Классическая теорема Квиллена-МакКорда для частично упорядоченных множеств формулируется следующим образом:\\

Пусть $PSet$ --- категория, объектами в которой являются классы изоморфизмов частично упорядоченных множеств (чумов), а морфизмами --- отображения, сохраняющие порядок, то есть такие $f :: X \to Y$, что $x_1 \leq x_2 \implies f(x_1) \leq f(x_2)$. Несложно убедиться, что это категория, это следует из рефлексивности и транзитивности частичного порядка.\\
Определён функтор $\mathcal{K} :: PSet \to Top$ --- функтор геометрической реализации, отображающий чум в его порядковый комплекс, то есть в симплициальный комплекс, каждый симплекс которого является линейной оболочкой векторов, формально сопоставленных элементам непустой цепи чума, а морфизм чумов в индуцированное им непрерывное отображение, совпадающее с ним на вершинах.\\
Пример действия функтора на объекте --- линейный порядок $a \leq b \leq c \leq d$ отображается в правильный тетраэдр $abcd$, при этом подпорядок $a \leq b \leq d$ отображается в грань $abd$ и так далее.\\
Обозначим через $U_x$ конус элемента $x \in X$, то есть $U_{x_0} = {x \in X | x \leq x_0}$.\\

\begin{theorem}
  Пусть $X, Y \in \operatorname{Ob}(PSet)$, $f \in \operatorname{Hom}(X,Y)$.\\
  Если $\forall y \in Y\;\mathcal{K}(f^{-1}(U_y))$ стягиваем, $\mathcal{K}f$ является гомотопической эквивалентностью.\\
\end{theorem}

\begin{itshape}
Эта теорема является наиболее естественным частным случаем теорема Квиллена $A$, устанавливающей условия, при которых морфизм геометрических реализаций нервов категорий $\mathcal{A}$ и $\mathcal{B}$ (т.н. классифицирующих пространств категорий), индуцированный функтором $\mathcal{F} :: \mathcal{A} \to \mathcal{B}$, является го*мотопической эквивалентностью.\\
В рассматриваемом случае $\mathcal{A}, \mathcal{B}$ --- чумы со структурой категории, в которой каждому элементу сопоставлен объект, а множество морфизмов между двумя объектами пусто, если соответствущая пара элементов не входи в порядок, и состоит из одного элемента иначе.\\ \\ $\mathcal{F}$ --- функтор, индуцируемый отображением $f$.\\
Общий случай расматриваться и доказываться в рамках этой работы не будет.
\end{itshape}

Верна также гомологическая версия теоремы:
\begin{theorem}
  Пусть $X, Y \in \operatorname{Ob}(PSet)$, $f \in \operatorname{Hom}(X,Y)$.\\
  Если $\forall y \in Y\;\mathcal{K}(f^{-1}(U_y))$ ацикличен, $\mathcal{K}f$ является гомологической эквивалентностью.\\
\end{theorem}

Доказательство теорем 1 и 2 будет в необходимом для формулировки результатов объёме дано согласно [1] в разделе 5.\\

В недавних исследованиях в области топологического анализа данных формулируется <<динамический>> вариант теории гомологий, гомологии модулей устойчивости. Цель этой работы --- сформулировать и доказать версию теоремы Квиллена-МакКорда для гомологий модулей устойчивости (обычно называемых устойчивыми гомологиями). Этот результат имеет также потенциальное практическое применение --- понижение в расчётных приложениях размерности нервов покрытий, полученных экспериментально или в результате работы алгоритма, для уменьшения их вычислительной сложности.

\subsection*{3. Модули устойчивости и формулировка теоремы}
Рассмотрим кольцо многочленов $\mathbb{K}[t]$. При перемножении многочленов их степени складываются, значит, на этом кольце существует градуировка степенями, то есть моноидом неотрицательных целых чисел по сложению. $\mathbb{K}[t] = \bigoplus_{i \in \mathbb(Z)_{\leq 0}} \{f \in \mathbb{K}[t] | deg(f) = i\} = \mathbb{K}[t]_i$. В дальшейшем при упоминании кольца многочленов мы будет предполагать наличие такой градуировки.\\
\begin{definition}
  Модуль устойчивости --- градуированный модуль над градуированным кольцом многочленов $\mathbb{R}[t]$.\\
  Градуировка целыми числами задана следующим образом: $M = \bigoplus_{j \in \mathbb{Z}} M^j:\; \forall i \in \mathbb(Z)_{\leq 0}\, \mathbb{R}[t]_i \cdot M^j \in M^{j+i}$.
\end{definition}
\begin{definition}
  Пусть $M$, $N$ --- модули устойчивости, $f :: M \to N$ --- отображение такое, что $\forall j\, f(M^j) \in f(N^{j+m})$ для некоторого $m \in \mathbb(Z)_{\leq 0}$. Такое отображение будем называть $m$-морфизмом, в частности, 0-морфизм --- морфизм в категории градуированных $\mathbb{R}$-модулей.
\end{definition}
Пусть $M$ --- модуль устойчивости. Обозначим как $Id_m :: M \to M$ существующий для любого модуля $m$-морфизм $f(a) = t^ma$.\\
\begin{definition}
  Пусть $M$ и $N$ --- модули устойчивости, $m \in \mathbb(Z)_{\leq 0}$, $f :: M \to N$, $g :: N \to M$ --- $m$-морфизмы. Если $f \circ g = Id_{2m}$ и $g \circ f = Id_{2m}$, модули $M$ и $N$ называются $m$-перемежёнными, $f$ и $g$ называются $m$-перемежениями.
\end{definition}
\begin{definition}
  Приближённым отношением эквивалентности будем называть бинарное отношение со следующими свойствами:
  \begin{itemize}
    \item $\forall M,m:\; M \stackrel{m}{\sim} M$ --- рефлексивность;\\
    \item $\forall M,N,m:\; M \stackrel{m}{\sim} N \iff N \stackrel{m}{\sim} M$ --- симметричность;\\
    \item $\forall M,N,K,m_1,m_2:\; M \stackrel{m_1}{\sim} N \land N \stackrel{m_2}{\sim} K \implies M \stackrel{m_1 + m_2}{\sim} K$ --- приближённая транзитивность.\\
  \end{itemize}
\end{definition}
FIXME: нужно ли нам обобщение для непрерывного времени?
\begin{claim}
  $m$-перемежённость --- приближённое отношение эквивалентности.
\end{claim}
\begin{proof}
  Рефлексивность и симметричность очевидны из определения.\\
  Приближённая транзитивность реализуется композицией реализующих отношение морфизмов $f_i, g_i$:\\
  $(f_2 \circ f_1) \circ (g_1 \circ g_2) = f_2 \circ (Id_{m_1}) \circ g_2 = Id_{m_1} \circ (Id_{m_2}) = Id_{m_1 + m_2}$, рассуждение в другую сторону симметрично.
\end{proof}
\begin{definition}
  Назовём $m$-тривиальным модуль, $m$-перемежённый с тривиальным модулем.
\end{definition}
\begin{lemma}
  Пусть $0 \to M \xrightarrow{i} N \xrightarrow{p} P \to 0$ --- короткая точная последовательность, то есть $\operatorname{Ker}p = \operatorname{Im}i$.\\
  Тогда если $M$ $m_1$-тривиален и $P$ $m_2$-тривиален, $N$ $m_1 + m_2$-тривиален.
\end{lemma}
\begin{proof}
  FIXME: diagram chasing.
\end{proof}
\begin{lemma}
  Пусть $0 \to M \xrightarrow{i} N \xrightarrow{p} P \to 0$ --- короткая точная последовательность.\\
  Тогда если $N$ $m$-тривиален, $M$ и $P$ $m$-тривиальны.
\end{lemma}
\begin{proof}
  FIXME: рассуждение на уровне элементов.
\end{proof}
\begin{definition}
  Топологическое пространство $X$ называется $m$-ацикличным, если $n \in \mathbb{Z}_{\geq 0} H_n(X,\mathbb{R})$ $m$-тривиален.
\end{definition}
\begin{claim}
  Есть топологическое пространство $X$ $m$-ациклично, его гомологии с любыми коэффициентами $m$-тривиальны.
\end{claim}
\begin{proof}
  FIXME: вообще не знаю, верно ли это, надо попробовать применить формулу универсальных коэффициентов и посмотреть на кручение. В общем, это не нужно.
\end{proof}
\begin{definition}
  Если отображение топологических пространств $f :: X \to Y$ индуцирует $m$-перемежения на всех модулях гомологий с коэффициентами в $\mathbb{R}$, $f$ называется гомологической $m$-эквивалентностью.
\end{definition}
\begin{theorem}
  \textbf{Приближённая версия теоремы 2.}\\
  Пусть $X, Y \in \operatorname{Ob}(PSet)$, $f \in \operatorname{Hom}(X,Y)$.\\
  Если $\forall y \in Y\;\mathcal{K}(f^{-1}(U_y))$ $m$-ацикличен, $\mathcal{K}f$ является гомологической $m$-эквивалентностью.\\
\end{theorem}
\subsection*{4. Устойчивость в спектральных последовательностях}

\subsection*{5. Доказательство теорем 1 и 2 по Бармаку}

\subsection*{6. Доказательство теоремы 3}

\subsection*{7. Использованная литература}
\begin{enumerate}
  \item Jonathan Ariel Barmak. 2011. On Quillen’s Theorem A for posets. J. Comb. Theory Ser. A 118, 8 (November 2011), 2445–2453. DOI:https://doi.org/10.1016/j.jcta.2011.06.008
  \item FIXME: ссылка на сербов.
\end{enumerate}

\end{document}
