We are now able to rely on apparatus of interleavings, hence are ready to establish the target theorem.\\

\begin{theorem} \textbf{Approximate Quillen-McCord theorem, draft statement}\\
  Assume $X, Y$ are persistence posets of finite type indexed by $I$, $f : X \to Y$ is order-preserving map, and one of the following holds:\\
  \begin{enumerate}
    \item $R$ is a field, $I = \mathbb{Z}$
    \item $R = \mathbb{Z}$, $I$ is an infinite additive subgroup in $\mathbb{R}$
  \end{enumerate}.

  Then if $\forall y=(\ldots,y_i,\ldots) \in Y\;\mathcal{B}(f^{-1}(Y_{\leqslant y}))$ is $\varepsilon$-acyclic over $R$, $\mathcal{B}f$ induces $e$-interleavings of all homology spaces over $R$ on $BX$ and $BY$.\\
\end{theorem}

Value of $e$ is set later.\\

\begin{proposition}
  Let $A$ and $B$ be two persistence topological spaces with at least one of them being $\varepsilon$-acyclic over $\mathbb{F}$. Then $A \star B$ is $\varepsilon$-acyclic over $\mathbb{F}$.
\end{proposition}

\begin{pf}
  Over a field all $Tor$-functors from Proposition 9 vanish since every vector space is free module, hence flat. Hence right hand side of expression of Proposition 9 is $\varepsilon$-trivial by Proposition 10.
\end{pf}

\begin{proposition}
  Let $0 \to \ker{f} \hookrightarrow A \xrightarrow{f} B \xrightarrow{\phi} C \xrightarrow{g} D \to \operatorname{coker}g \to 0$ be exact sequence in category of non-negatively graded vector spaces. Then if $A$ and $D$ are $\varepsilon$-trivial, then $B \stackrel{2\varepsilon}{\sim} C$ with $\phi$ being left morphism in interleaving pair.
\end{proposition}

\begin{pf}
  In s.e.s $0 \to \ker{f} \hookrightarrow A \xrightarrow{f} \operatorname{im}f \to 0$ $\operatorname{im}f$ is $\varepsilon$-trivial. By exactness it is equal to $K = \ker \phi$. On the other side from $0 \to \ker{g} \xrightarrow{g} D \to D \to 0$ there follows that $I = \operatorname{im} \phi$ is $\varepsilon$-trivial. Hence by transitivity $K \stackrel{2\varepsilon}{\sim} I$.\\

  We obtain exact sequence $0 \to K \to B \xrightarrow{\phi} C \to I \to 0$. $B = K \oplus \operatorname{coIm}\phi$, $C = I \oplus \operatorname{Im} \phi$. Coimage and image are canonically isomorphic, hence by Proposition 10 proposition follows.
\end{pf}\\

\begin{definition} ~ \par
  Consider poset $(X,\phi)$ and sets $Y_i \subset X_i$.\\

  If there is no element $y$ in any $Y_i$ such that $\phi(y) \not\in Y_{i+1}$ then component-wise inclusion $Y \to X$ commutes with structure maps and form embedding of persistence posets. In this case $Y=(\ldots,Y_i,\ldots)$ is a \define {persistence subposet} of $X$.
\end{definition}

\begin{definition}
  Assume poset $X$ splits into subposets $X_j$. It is not a trivial assumption since arbitrary split can lose some structure maps. Then every subposet $X_j$ has its own classifying space $BX_j$. If these spaces (or minimal open sets containing them) cover the whole $BX$, they are called \define{persistence covering}.
\end{definition}

Let $X=(\ldots,X_i,\ldots)$ be persistence poset. Take arbitrary $x_i \in X_i$ where possible, and special bottom element $x_i = \bot$ for empty $X_i$'s. We call $x = (\ldots, x_i, \ldots)$ element of $X$. Sets of type $X_{<x}$ with different relations to $x$ are all defined component-wise and are all empty in components with $x_i = \bot$.\\

\begin{proposition}
  For $x = (\ldots,x_i,\phi(x_i),\ldots)$ with all $x_i$ except firsts after bottoms being images of structure maps covering from Remark 6 can be extended to persistence covering $\mathcal{U}$ with $Y^1$ --- preimage of $st(\mathcal{N}(x))$ under nerve functor and $Y^2 = X \setminus \{x\}$.
\end{proposition}

\begin{pf}
  It suffices to check that $X \setminus \{x\}$ and preimage of $st(\mathcal{N}(x))$ are subposets.\\

  It is evident for $X \setminus \{x\}$. Elements in the preimage of $\operatorname{st}(\mathcal{N}(x_i))$ are exactly elements comparable to $x_i$. Since structure maps preserve order, they do not move comparable elements to incomparable. Hence preimage also forms subposet.
\end{pf}\\

\begin{lemma}
  Let $(X,\phi)$ be a persistence poset and for $x=(\ldots,x_i,\phi(x_i),\ldots) \in X$ either $\mathcal{B}(X_{< x})$ or $\mathcal{B}(X_{> x})$ is $\varepsilon$-acyclic. Then embedding $\mathcal{B}(X \setminus \{x\}) \hookrightarrow \mathcal{B}(X)$ induces $2\varepsilon$-interleavings of all homology spaces.
\end{lemma}

\begin{pf} ~ \par
  By Proposition 11 $\mathcal{K}(\operatorname{lk}(\mathcal{N}(x)))$ is $\varepsilon$-acyclic.\\

  Given persistence covering we have persistence embeddings of covering sets into $X$. In homology we have induced maps. Category of persistence modules is abelian with kernels and cokernels computable pointwise.\\

  Hence we can define persistence Mayer-Vietoris exact sequence on homology spaces component-wise by gluing sequences for components over structure maps. This is an exact sequence in $Fun(I,R-Mod)$. By equivalence of categories it defines exact sequence $Seq$ of non-negatively graded $R[I]$-modules.\\

  $Seq$ construction on $\mathcal{U}$ and Proposition 12 yield the lemma.
\end{pf}

\begin{definition}
  We will denote as \define{linear extension of persistence poset} $X$ series of linear extensions of $X_i$ such with the same structure maps as $X$.
\end{definition}

\begin{proposition}
  Every persistence poset has linear extension.
\end{proposition}

\begin{pf}
  Let $X$ be persistence poset with structure maps $\phi$. We will denote map from $X_i$ as $\phi_i$ and refer to composition of $k$ structure maps as to $\phi^k$ to simplify notation.\\

  For any incomparable pair $a,b$ in $X_i$ one of the following hold:
  \begin{enumerate}
    \item $a \in \operatorname{im}(\phi_{i-1})$, $b \in \operatorname{im}(\phi_{i-1})$. Then we know that preimages are incomparable since $\phi_{i-1}$ is order preserving.
    \item $a \in \operatorname{im}(\phi_{i-1})$, $b \not\in \operatorname{im}(\phi_{i-1})$ or vice versa.
    \item Both $a$ and $b$ are not in image of $\phi_{i-1}$.
  \end{enumerate}

  Assume for some incomparable $a,b$ in $X_i$ $\phi^k(a)$ and $\phi^k(b)$ are comparable for some natural $k$. W.l.o.g $\phi^k(a) \leq \phi^k(b)$. Then we have to add pair $(a,b)$ to order on $X_i$. There is no possible contradiction since pair $(b,a)$ in order implies $\phi^k(b) \leq \phi^k(a)$.\\
  This extension extends infinitely to the left since we never have ordered pair in iterated preimages of $a$ and $b$. If there is no such $k$ we leave ordering on $a,b$ to order extension principle. This extension can be safely extended to both sides.\\

  Application of this reasoning to all poset components and incomparable pairs in them yields proposition.
\end{pf}\\

\begin{remark}
  While working with persistence topological spaces we sometimes see properties which hold component-wise. If for persistence space $X$ or map $f_i$ there is a property which holds for any $X_i$ or $f_i$ we call it component-wise property of $X$. For instance, if map $f$ is component-wise homotopy equivalence, it induces $0$-interleaving of homology modules.
\end{remark}

We are now ready to adapt known proof to Quillen-McCord theorem for persistence posets.
%Case of coefficients in $\mathbb{Z}$ is better to be considered separately.

\begin{theorem} \textbf{Approximate Quillen-McCord theorem, final statement}\\
  Assume $X, Y$ are persistence posets of finite type indexed by $I$, $f : X \to Y$ is order-preserving map. Let $m = \max_{i}(|Y_i|)$ --- maximal cardinality of components of $Y$ and one of the following holds:\\
  \begin{enumerate}
    \item $R$ is a field, $I = \mathbb{Z}$
    \item $R = \mathbb{Z}$, $I$ is an infinite additive subgroup in $\mathbb{R}$
  \end{enumerate}.

  Then if $\forall y=(\ldots,y_i,\ldots) \in Y\;\mathcal{B}(f^{-1}(Y_{\leqslant y}))$ is $\varepsilon$-acyclic over $R$, $\mathcal{B}f$ induces $2m\varepsilon$-interleavings of all homology spaces over $R$ on $BX$ and $BY$.\\
\end{theorem}

\begin{pf} \textbf{Approximate Quillen-McCord theorem}\\
  Let $X, Y$ be persistence posets of finite type with order-preserving map $f : X \to Y$.\\

  Let $\overline{X}$ be linear extension of $X$. Let $x^i_1, x^i_2, \ldots, x^i_{n_i}$ be enumeration of $\overline{X}_i$ in its linear order and $Y_i^r = \{x^i_1,\ldots,x^i_r\} \cup Y_i \subset M(f_i)$ for any $r$ up to $\max_{i}(n_i)$ and $i$. $Y^r = (\ldots, Y_i^r, \ldots)$ is a persistence subposet of $M(f)$ with extended order on $X$.\\

  There are posets such that $n_i < r$. In these cases notation $x_r$ means that on positions with $n_i < r$ we take $x_{n_i}$. $n_i$ can be undefined, in this case $x_r = \bot$.\\

  Consider $Y^r_{>x_r} = Y_{\geq f(x_r)}$. $\mathcal{B}(Y_{\geq f(x_r)})$ is a component-wise cone over $\mathcal{B}(f(x_r))$. It is component-wise contractible, therefore $\mathcal{B}(Y^{r-1}) \hookrightarrow \mathcal{B}(Y^{r})$ is component-wise homotopy equivalence by lemma 2. By iteration $\mathcal{B}(j) : \mathcal{B}(Y^{0}) = \mathcal{B}(Y) \hookrightarrow \mathcal{B}(M(f)) = \mathcal{B}(Y^n)$ is component-wise homotopy equivalence between $BY$ and $M(f)$. Note that persistence structure is not used here.\\

  Then consider linear extension of $Y$ with enumerations $y^i_1,\ldots,y^i_{m_i}$ and $X_i^r = X \cup \{y^i_{r+1},\ldots,y^i_{m_i}\} \subset M_i(f)$. $X^{r-1}_{\leq y_r} = f^{-1}(Y_{\leqslant y_r})$. Latter is $\varepsilon$-acyclic over $\mathbb{F}$ by condition of the theorem. Hence $\mathcal{B}(X^{r}) \hookrightarrow \mathcal{B}(X^{r-1})$ induces $2\varepsilon$-interleavings of all homology modules and by transitivity of $\varepsilon$-equivalence $\mathcal{B}(i)$ induces $2m\varepsilon$-interleavings between homology of $X$ and $M(f)$.\\

  Note that $i(x) \leqslant (j \circ f)(x)$. By Proposition 7 $\mathcal{B}(i)$ is component-wise homotopic to $\mathcal{B}(j \circ f) = \mathcal{B}(j) \circ \mathcal{B}(f)$. Homotopic maps induce same maps on homology, $j$ is a homotopy equivalence and induce $0$-interleavings. Hence $\mathcal{B}(f)$ induce $2m\varepsilon$-interleavings between $H_i(BX,R)$ and $H_i(BY,R)$.
\end{pf}
