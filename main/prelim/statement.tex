\begin{definition}
  \define{Join} $A \star B$ of simplicial complexes $A$ and $B$ is the simplicial complex with simplices --- all possible unions of simplices $a \in A$ and $b \in B$.
\end{definition}

\begin{definition}
  \define{Join} of topological spaces $A$ and $B$ is defined as follows: $A \star B\ := A \sqcup_{p_0} (A \times B \times [0,1]) \sqcup_{p_1} B$, where $p$ are projections of the cylinder $A \times B \times [0,1]$ onto faces.
\end{definition}

\begin{definition}
  \define{Star} $\operatorname{st}(x)$ of simplex $x \in A$ ($A$ is a simplicial complex) is the minimal simplicial complex containing all simplices $a \in A$ such that there exists inclusion $x \hookrightarrow a$.
\end{definition}

\begin{definition}
  \define{Link} $\operatorname{lk}(x)$ of simplex $x \in A$ is defined as follows: $\operatorname{lk}(x) = \{v \in \operatorname{st}(x)|\; x \not\in v\}$.
\end{definition}

\begin{definition}
  Let $\Delta$ be a simplicial category. Functor $\left|\right| : \Delta \to Top$ which maps simplices to geometric simplices of corresponding dimension and morphisms to inclusions of faces and restrictions to subcomplexes is called standard \define{geometric realization}.
\end{definition}

\begin{proposition}
  $\operatorname{st}(x) = \operatorname{lk}(x) \star x$.
\end{proposition}

\begin{proposition}
  $\left|A \star B\right| = \left|A\right| \star \left|B\right|$. Hence $\left|\operatorname{st}(x)\right|$ is a cone over $\left|x\right|$.
\end{proposition}

Let $\mathcal{C}$ denote a small category. We can construct the simplicial set called the \define{nerve of category $\mathcal{C}$} as follows:\\
Let objects of $\mathcal{C}$ be the only 0-dimensional simplices. Then let all morphisms be 1-dimensional simplices, all composable pairs of morphisms be 2-dimensional simplices and so on with morphisms --- inclusions of compositions into longer ones and replacements of compositions $i \circ j$ with $f = i \circ j$.\\
This construction is functorial over category of posets $Pos$, we denote nerve functor from $Pos$ to category of simplicial sets over $\Delta$ as $\mathcal{N}$.

\begin{definition}
  Geometric realization $B\mathcal{C}$ of $\mathcal{N}(\mathcal{C})$ is called \define{classifying space} of $\mathcal{C}$.\\

  We denote composition of nerve and geometric realization as $\mathcal{B}$. It is obviously a functor and we only need this construction to avoid confusing notation $Bf$ for induced morphism of classifying spaces. By definition $\mathcal{B}(\mathcal{C}) = B\mathcal{C}$.
\end{definition}

\begin{definition}
  Let $f: \mathcal{C} \to \mathcal{D}$ be a functor and $d$ --- object in $\mathcal{D}$. Then \define{comma category} $d \downarrow f$ is a category with objects --- pairs $(s,i_s)$ of objects in $\mathcal{C}$ and morphisms $i_s : d \to f(s)$ and morphisms --- morphisms $g$ in $\mathcal{C}$ such that triangle $i_s, i_{g(s)}, f(g)$ is commutative.
\end{definition}

\begin{theorem} {\cite[Theorem A]{Quillen72}}\\
  If $f: C \to D$ is a functor such that the classifying space $B(d \downarrow f)$ of the comma category $d \downarrow f$ is contractible for any object $d \in D$, then $f$ induces a homotopy equivalence $BC \to BD$.
\end{theorem}

Nerve construction on a poset category yields a simplicial complex called \define{order complex}. Application of Quillen A theorem to posets yields the following theorem (we identify poset with its poset category):

\begin{theorem} \textbf{Quillen-McCord theorem}\\
  Assume $X, Y$ are finite posets, $f : X \to Y$ is order-preserving map.\\
  If $\forall y \in Y\;\mathcal{B}(f^{-1}(Y_{\leqslant y}))$ is contractible, then $\mathcal{B}f$ is a homotopy equivalence between $BX$ and $BY$.\\
\end{theorem}

Proof of the theorem is necessary for construction of a desired result.\\
