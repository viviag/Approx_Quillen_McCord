There holds homological {\cite[Corollary 5.5]{Bar11}} versions of this theorem:

\begin{theorem} \textbf{Homological Quillen-McCord theorem}\\
  Assume $X, Y$ are finite posets, $f : X \to Y$ is order-preserving map, $R$ is a PID.\\
  If $\forall y \in Y\;H_i(\mathcal{B}(f^{-1}(Y_{\leqslant y})),R) = 0$ for any $i$, $\mathcal{B}f$ induces isomorphisms of all homology groups with coefficients in $R$ on $BX$ and $BY$.\\
\end{theorem}

To derive the theorem we can variate the proof of standard version in the following manner:\\

\begin{proposition} {\cite[Lemma 2.1]{Milnor56}}
  \label{prop:kunneth}
  Reduced homology modules with coefficients in a principal ideal domain of a join satisfy the relation
  $H_{r+1}(A \star B, R) \simeq \bigoplus_{i+j=r}(H_i(A,R) \otimes_R H_j(B,R)) \oplus \bigoplus_{i+j=r-1} \operatorname{Tor_1^R}(H_i(A,R),H_j(B,R))$.
\end{proposition}

\begin{lemma}
  \label{lem:trivialties}
  Let $X$ be a finite poset and for $x \in X$ either $H_i(\mathcal{B}(X_{< x}))$ or $H_i(\mathcal{B}(X_{> x}))$ with coefficients in a PID are equal to homology of a point. Then embedding $\mathcal{B}(X \setminus \{x\}) \hookrightarrow \mathcal{B}(X)$ induces isomorphisms of all homology groups.
\end{lemma}

\begin{pf} ~ \par
  By Proposition \ref{prop:kunneth} $H_i(\mathcal{K}(\operatorname{lk}(\mathcal{N}(x)))) = H_i(\mathcal{B}(X_{>x}) \star \mathcal{B}(X_{<x}))$ are trivial for all indices $i$ --- $Tor$-functors vanish if any of their arguments is trivial.
  Application of Mayer-Vietoris long exact sequence to covering from Remark \ref{covering} yields the lemma.
\end{pf}\\

Proof of the theorem is similar to proof finishing previous section. However, we write it here in detail in order to be able to highlight differences. Changed parts are written in italic.\\

\begin{pf} \textbf{Homological Quillen-McCord theorem}\\
Let $X, Y$ be finite posets with order-preserving map $f : X \to Y$.\\

Every poset has linear extension. Let $x_1, x_2, \ldots, x_n$ be enumeration of $X$ in such linear order and $Y^r = \{x_1,\ldots,x_r\} \cup Y \subset M(f)$ for any $r$.\\

Consider $Y^r_{>x_r} = Y_{\geq f(x_r)}$. $\mathcal{B}(Y_{\geq f(x_r)})$ is a cone over $\mathcal{B}(f(x_r))$. It is contractible, therefore $\mathcal{B}(Y^{r-1}) \hookrightarrow \mathcal{B}(Y^{r})$ is homotopy equivalence by lemma \ref{lem:homotopy}. By iteration $\mathcal{B}(j) : \mathcal{B}(Y^{0}) = \mathcal{B}(Y) \hookrightarrow \mathcal{B}(M(f)) = \mathcal{B}(Y^n)$ is homotopy equivalence between $BY$ and $M(f)$.\\

Then consider linear extension of $Y$ with enumeration $y_1,\ldots,y_m$ and $X^r = X \cup \{y_{r+1},\ldots,y_n\} \subset M(f)$. $X^{r-1}_{\leq y_r} = f^{-1}(Y_{\leqslant y_r})$. \textit{Latter is acyclic over $R$ by condition of the theorem. Hence $\mathcal{B}(X^{r}) \hookrightarrow \mathcal{B}(X^{r-1})$ induces isomorphisms of all homology groups and by functoriality of homology $\mathcal{B}(i)$ induces isomorphisms of all homology groups between $X$ and $M(f)$.}\\

Note that $i(x) \leqslant (j \circ f)(x)$. By Proposition \ref{prop:comparison} $\mathcal{B}(i)$ is homotopic to $\mathcal{B}(j \circ f) = \mathcal{B}(j) \circ \mathcal{B}(f)$. \textit{Homotopic maps induce same maps on homology, $j$ is a homotopy equivalence and induce isomorphisms. Hence $\mathcal{B}(f)$ induce isomorphisms between $H_i(BX,R)$ and $H_i(BY,R)$.}
\end{pf}\\

We see two updates. First one is essential, it requires Lemma \ref{lem:trivialties} and operates some equivalence propagating in a chain of length equal to cardinality of $Y$. Second follows automatically from functoriality of all used constructions.
