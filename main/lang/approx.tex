Persistence modules over good enough ring (as always here and after) give us the first example of approximation distance. So we can infer some results about distances.\\

\begin{proposition} ~ \par
  \label{prop:sum}
  Let $A$, $B$ be two persistence modules such that $d(A,0) \leq \varepsilon$ and $d(B,0) \leq \varepsilon$. Then $d(A \oplus B,0) \leq \varepsilon$.
\end{proposition}

\begin{pf}
  $\alpha A \oplus \alpha B = \alpha(A \oplus B)$. Result follows by Proposition $\ref{epstriv}$ via Theorems 1 and 2.
\end{pf}

\begin{proposition} ~ \par
  \label{prop:tensor}
  Let $A$, $B$ be two persistence modules such that $d(A,0) \leq \varepsilon$ and $d(B,0) \leq \varepsilon$. Then $d(A \otimes B,0) \leq \varepsilon$.
\end{proposition}

\begin{pf}
  Result follows from bilinearity of tensor product and Theorems 1 and 2.
\end{pf}

\begin{proposition}
  \label{prop:hominter}
  Let $P = \ldots \to P_n \to P_{n-1} \to \ldots$ be a persistence complex such that for all $i$'s $P_i \stackrel{\varepsilon}{\sim} 0$. Then the homology modules of $P$ are $\varepsilon$-interleaved with $0$.
\end{proposition}

\begin{pf}
  Assume $d_i$ is a differential in a complex. We know that $0 \to \operatorname{im}{d_{i+1}} \to \ker{d_{i}} \to H_i(P) \to 0$ is exact and that $0 \to \ker{d_i} \to P_i \to P_{i-1} \to 0$ is exact. Result follows by application of Lemma 1 twice.
\end{pf}\\

\begin{proposition} {\cite[Page 2]{Mitchell81}}\\
  Category $Fun(I, $R$-Mod)$ has enough projectives.
\end{proposition}

Since we have enough projectives we can compute derived functors. We need the following proposition.\\

\begin{proposition}
  \label{prop:tor}
  Let $R$ be commutative ring, $A$ and $B$ --- $R$-modules such that either $A$ or $B$ is $\varepsilon$-interleaved with $0$. Then $Tor_i^R(A,B) \stackrel{\varepsilon}{\sim} 0$.
\end{proposition}

\begin{pf}
  Since $R$ is commutative, $Tor_i^R(A,B) = Tor_i^R(B,A)$. Without loss of generality assume $B \stackrel{\varepsilon}{\sim} 0$. Let $P$ be the projective resolution of $A$. After taking tensor product we obtain by Proposition \ref{prop:tensor} sequence of modules $\varepsilon$-interleaved with $0$. Proposition follows by Proposition \ref{prop:hominter}.
\end{pf}\\

We can also derive the result about exact sequences.\\

\begin{proposition}
  \label{major}
  Let $A \xrightarrow{f} B \xrightarrow{\phi} C \xrightarrow{g} D$ be exact sequence in category of persistence modules. Then if $d(A,0) \leq \varepsilon$ and $d(D,0) \leq \varepsilon$, then $B \stackrel{4\varepsilon}{\sim} C$ with $\phi$ being left morphism in interleaving pair.
\end{proposition}

\begin{pf}
  Under conditions of theorems 1 and 2 we can identify the category of persistence modules and the category of non-negatively graded modules.\\

  In s.e.s $0 \to \ker{f} \hookrightarrow A \xrightarrow{f} \operatorname{im}f \to 0$ $\operatorname{im}f$ is $\varepsilon$-trivial. By exactness it is equal to $K = \ker \phi$. On the other side from $0 \to \ker{g} \xrightarrow{g} D \to D \to 0$ there follows that $d(I = \operatorname{im} \phi, 0) \leq \varepsilon$. Hence by transitivity $K \stackrel{2\varepsilon}{\sim} I$.\\

  We obtain exact sequence $0 \to K \to B \xrightarrow{\phi} C \to I \to 0$. This sequence decomposes into sequences $0 \to K \to B \to \operatorname{coIm}\phi$ and $0 \to \operatorname{Im}\phi \to C \to I \to 0$. By lemma \ref{ops} we have that $d(B,\operatorname{coIm}\phi) \leq 2\varepsilon$ and $d(C,\operatorname{Im}\phi) \leq 2\varepsilon$. $coIm$ and $Im$ are pointwise canonically isomorphic by first isomorphism theorems for modules, hence $d(B,C) \leq 4\varepsilon$.
\end{pf}\\
