In his proof of Quillen-McCord theorem Barmak relies on order extension principle. To be able to transfer Barmak's proof of Quillen-McCord theorem to persistent case we have to stress similar statement for persistence posets.

\begin{definition}
  We denote as \define{extension of persistence poset} $X$ series of partially-ordered extensions of $X_i$ such that structure maps of $X$ are well-defined on these extensions. If extensions of all components are linear we call this series linear extension.
\end{definition}

\begin{proposition}
  \textbf{Transfer of order}. Let $f$ be a morphism between posets $X$ and $Y$ and $\overline{Y}$ be a linear extension of $Y$. Then $f$ induces partially ordered extension $\hat{X}$ of $X$ such that $f$ is well-defined as map $\hat{X} \to \overline{Y}$.
\end{proposition}

\begin{pf}
  Consider two incomparable points $a, b \in X$ and map $f : X \to \overline{Y}$ which is obviously well-defined.\\
  One of the following hold:\\
  \begin{itemize}
    \item f(b) < f(a)
    \item f(a) < f(b)
    \item f(a) = f(b)
  \end{itemize}
  If strict inequality holds, we can impose a single relation on $a$ and $b$ --- we inherit relation from images.\\
  If equality holds we do not.
\end{pf}

\begin{proposition}
  \textbf{Left propagation of linear extension.} Assume indexing set $I$ of persistence poset $(X,\phi)$ is well-founded with respect to inverted order and there exists maximal index $i$ such that $X_j = \emptyset$ for any $j > i$. Then $(X,\phi)$ has linear extension.
\end{proposition}

\begin{pf}
  We can extend order on the component $X_i$ to linear. Given this order we can transfer it to the left via all possible structure maps by proposition. We obtain extension of $(X,\phi)$ because all preimages of incomparable elements were imcomparable and we have equipped them with compatible orders. Now let's assume we obtained by this construction linear orders in components $X_j$ for $j > j_0$. Then $X_{j_0}$ can be linearly extended. Proposition follows by transfinite induction.
\end{pf}

This statement can also be seen as a corollary of a more general proposition.
One can consider set $E(X,\phi)$ of extensions of persistence poset $(X,\phi)$ with partial order defined as follows: let $Y,Z$ be extensions of $X$, then $Y \geq Z$ if and only if $Y$ is extension of $Z$. Since underlying sets of these extensions are always the same we can identify element of $E(X,\phi)$ with tuple of order relations on components of $X$.

\begin{proposition}
  Every linearly ordered subset of $E(X,\phi)$ has an upper bound in $E(X,\phi)$.
\end{proposition}

\begin{pf}
  Let $\{R^s|\;s \in S\}$ be a linearly ordered subset of $E(X,\phi)$ indexed by set $S$. Consider $R = \bigcup R^s$ where union is taken component-wise. Assume for some elements $a, b \in X_i$ for some $i$ some $\phi_{ij}$ cannot be defined on them as order-preserving map. Then there exists $s \in S$ such that $a$ and $b$ are comparable in extension $R^s$. But $R^s$ is an extension, hence $\phi_{ij}$ is defined on both $a$ and $b$. By contradiction proposition follows.
\end{pf}

\begin{proposition}
  \textbf{Persistent order extension principle.} Every persistence poset $(X,\phi)$ has linear extension.
\end{proposition}

\begin{pf}
  By Zorn's lemma $E(X,\phi)$ has maximal element $M$. Assume this element is not a linear extension of $(X,\phi)$.\\

  Then in some $M_i$ there exists incomparable pair $(a,b)$. Consider $\phi_{ij}(a)$ and $\phi_{ij}(b)$ for all $j > i$. Assume without loss of generality for $j_2 > j_1$ that $\phi_{ij_1}(a) > \phi_{ij_1}(b)$ and $\phi_{ij_2}(a) < \phi_{ij_2}(b)$. Then map $\phi_{j_1j_2}$ cannot be defined. Hence for any $j$ relation between images of $a$ and $b$ has the same sign if exists. If there exists such $j$ that relation between $\phi_{ij}(a)$ and $\phi_{ij}(b)$ exists we define relation between $a$ and $b$ accordingly. Otherwise we define it arbitrarily and propagate order to the left --- all preimages of $a$ and $b$ were incomparable and are now equipped with compatible orders.\\

  We have constructed an extension of $M$ which is taken to be maximal. Hence $M$ must be linear extension of $(X,\phi)$.
\end{pf}
