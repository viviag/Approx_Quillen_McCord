In his proof of Quillen-McCord theorem Barmak relies on order extension principle. To be able to transfer Barmak's proof of Quillen-McCord theorem to persistent case we have to stress similar statement for persistence modules.

\begin{definition}
  We denote as \define{linear extension of persistence poset} $X$ series of linear extensions of $X_i$ such that structure maps of $X$ are well-defined on these extensions.
\end{definition}

Our target proposition (which we shall refer to as \define{persistent order extension principle}) is the following: every persistence poset has linear extension.

\begin{proposition}
  \textbf{Transfer of linear order}. Let $f$ be a morphism between posets $X$ and $Y$ and $\overline{Y}$ be a linear extension of $Y$. Then $f$ induces at least one linear extension $\overline{X}$ of $X$ such that $f$ is well-defined as map $\overline{X} \to \overline{Y}$.
\end{proposition}

\begin{pf}
  Consider two incomparable points $a, b \in X$ and map $f : X \to \overline{Y}$ which is obviously well-defined.\\
  One of the following hold:\\
  \begin{itemize}
    \item f(b) < f(a)
    \item f(a) < f(b)
    \item f(a) = f(b)
  \end{itemize}
  If strict inequality holds, we can impose a single relation on $a$ and $b$ --- we inherit relation from images.\\
  If equality holds we can take arbitrary order on a set $f^{-1}(f(a))$.
\end{pf}

\begin{corollary}
  Every persistence poset $(X,\phi)$ with finite number of nonempty components has linear extension.
\end{corollary}

\begin{pf}
  We can extend order to linear on the component $X_i$ such that $X_j = \emptyset$ for $j > i$. Given this order we can extend it to the left by proposition.
\end{pf}

This corollary is strong enough to be used in a proof of the target theorem. Proof of this corollary allows also to reasons about sequences infinite to the left. For sequences infinite to the right we can use the observation that $Pos$ is cocomplete {\cite[Example 1.10 and Theorem 1.11, page 17]{Adamec94}}.

\begin{proposition}
  \textbf{Persistent order extension principle}. Every persistence poset over well-founded indexing set has linear extension.
\end{proposition}

\begin{pf}
  Let $(X,\phi)$ be a persistence poset. It is by definition a diagram in $Pos$. $Pos$ is cocomplete, hence there exists $\operatorname{colim}(X, \phi)$. In particular it is a co-cone over $(X,\phi)$ with maps $\psi_i : X_i \to \operatorname{colim}(X, \phi)$. For all $i$ $\psi_i$ transfers linear extension from $\overline{\operatorname{colim}(X, \phi)}$ to $X_i$.\\

  Consider co-cone diagram\\
  \begin{tikzcd}
    X_i \arrow{dr}{\psi_i}\arrow{rr}{\phi_{i}} & & X_{j}\arrow{dl}{\psi_{i+1}} \\
    & \operatorname{colim}(X, \phi) &  \\
  \end{tikzcd}
  It is commutative, i.e. $\psi_i = \psi_{i+1} \circ \phi_i$. Extend colimit linearly and transfer this extension by $\psi$. It's enough to check that $\phi_i$ is well-defined as a morphism between $\overline{X_i}$ and $\overline{X_{i+1}}$.\\

  Assume it is not. Then there exists pair of elements $a,b \in \overline{X_i}$ such that $a > b$ and $\phi_i(a) < \phi_i(b)$. There are three possibilities:
  \begin{itemize}
    \item $a > b$ is an element of order on $X_i$. It is impossible since $\phi_i$ preserves order on $X_i$.
    \item Relation $a > b$ is induced by relation $\psi_i(a) > \psi_i(b)$ on $\overline{\operatorname{colim}(X, \phi)}$. In this case from commutativity of the diagram we know that this relation induces relation $\phi_i(a) > \phi_i(b)$.
    \item $\psi_i(a) = \psi_i(b)$.
  \end{itemize}

  In last case we have to adjust our transferred extensions and make them less arbitrary. At first fix element $\alpha \in \overline{\operatorname{colim}(X, \phi)}$. In all components of $(X,\phi)$ order between preimages of $\alpha$ is not defined by extension of colimit. By diagram structure maps move preimages of $\alpha$ to preimages of $\alpha$, possibly not all. \textit{We identify $\alpha$ with its equivalence class.}\\

  For arbitrary $i \in I$ fix linear order on $\psi_i^{-1}(\alpha)$. Its back propagation is given by transfer along $\phi$, its forth propagation is given by co-cone diagram.\\

  % Should be discussed.
  So we fix $i \in I$, fix orders on $\psi_i^{-1}$ for all equivalence classes in colimit and perform an extension with both kinds of propagation. For any $i$ we have constructed persistence poset with $X_j$ linearly ordered for all $j \leq i$. $I$ is well-founded hence by transfinite recursion result follows.
\end{pf}

% \begin{pf}
%   Let $(X, \phi)$ be a persistence poset. We denote map from $X_i$ as $\phi_i$ and refer to composition of $k$ structure maps as to $\phi^k$ to simplify notation.\\
%
%   For any incomparable pair $a,b$ in $X_i$ one of the following hold:
%   \begin{enumerate}
%     \item $a \in \operatorname{im}(\phi_{i-1})$, $b \in \operatorname{im}(\phi_{i-1})$. Then we know that preimages are incomparable since $\phi_{i-1}$ is order preserving.
%     \item $a \in \operatorname{im}(\phi_{i-1})$, $b \not\in \operatorname{im}(\phi_{i-1})$ or vice versa.
%     \item Both $a$ and $b$ are not in image of $\phi_{i-1}$.
%   \end{enumerate}
%
%   Assume for some incomparable $a,b$ in $X_i$ $\phi^k(a)$ and $\phi^k(b)$ are comparable for some natural $k$. W.l.o.g $\phi^k(a) \leq \phi^k(b)$. Then we have to add pair $(a,b)$ to order on $X_i$. There is no possible contradiction since pair $(b,a)$ in order implies $\phi^k(b) \leq \phi^k(a)$.\\
%   This extension extends infinitely to the left since we never have ordered pair in iterated preimages of $a$ and $b$. If there is no such $k$ we leave ordering on $a,b$ to order extension principle. This extension can be safely extended to both sides.\\
%
%   Application of this reasoning to all poset components and incomparable pairs in them yields proposition.
% \end{pf}\\
