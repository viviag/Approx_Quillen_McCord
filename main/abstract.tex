Common approach in Topological Data Analysis is to compute homology groups of some persistence complex constructed from experimental data and try to classify some objects by them. Complexity of computation of persistent homology is high enough to limit practical usage of the approach.\\

It raises a question of how to construct persistence complex with same homology as initially given with smaller dimension. Some of the complexes arise from dynamics of some partially ordered set by nerve construction and Quillen-McCord theorem (Quillen fiber lemma) gives sufficient condition when map between posets induces a homotopy equivalence in classifying spaces of their poset categories. In this paper we prove Quillen-McCord theorem in the setting of persistent homology. We believe it to be the tool to reduce complexity of observed persistence complex.

While proving the theorem we develop a setting convenient to prove approximate statements about persistence objects (and define persistence object as object of appropriate functor category). This setting is based on notion of interleaving distance and its behavior under operations in category of persistence objects.
